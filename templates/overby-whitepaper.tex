% Overby Industries White Paper Template
\documentclass[12pt]{article}

% Packages
\usepackage{geometry}
\usepackage{graphicx}
\usepackage{xcolor}
\usepackage{hyperref}
\usepackage{fancyhdr}
\usepackage{titlesec}
\usepackage{tocloft}
\usepackage{amsmath}
\usepackage{amssymb}

% Page setup
\geometry{margin=1in}
\setlength{\parskip}{0.8em}
\setlength{\parindent}{0em}

% Header/footer
\pagestyle{fancy}
\fancyhf{}
\fancyhead[L]{Overby Industries White Paper}
\fancyhead[R]{\thepage}

% Title formatting
\titleformat{\section}{\large\bfseries}{\thesection}{1em}{}
\titleformat{\subsection}{\normalsize\bfseries}{\thesubsection}{1em}{}

% Hyperlinks style
\hypersetup{
    colorlinks=true,
    linkcolor=blue,
    urlcolor=blue,
    citecolor=blue
}

% Title Info
\title{
    \Huge \textbf{Overby Industries White Paper} \\[0.5em]
    \Large Towards Open, Democratic Spaceflight \\[1em]
    {\large Version 1.0 – Draft for Community Review}
}
\author{Overby Industries R\&D Division}
\date{\today}

\begin{document}

\maketitle
\thispagestyle{empty}

\begin{abstract}
This document introduces the Overby Industries planetary-scale power and propulsion framework. It details the Magnetohydrodynamic (MHD) Generator, Ionic-Liquid Thrusters, and Solar Wind Capture Systems that form the foundation of our closed-loop architecture. The purpose of this white paper is to share concepts openly, invite constructive input, and establish Overby Industries’ role as a civic aerospace movement.
\end{abstract}

\newpage
\tableofcontents
\newpage

\section{Introduction}
Space is not just a frontier, it is the future commons of humanity. Overby Industries aims to unlock practically unlimited $\Delta V$ by merging power generation and propulsion into one architecture.

\section{System Overview}
\subsection{MHD Ionic-Liquid Generator}
Describe chamber design, electron arcs, coils, and dual role as generator + thruster.

\subsection{Ionic Liquid Chemistry}
Outline ionic propellants (e.g., EMIMBF4, AlCl3-Urea), nonflammability, capacitive synergy.

\section{Mission Architecture}
Summarize staged deployment: Earth → Moon → Mars → Asteroids → Titan → Deep Space.

\section{Conclusion}
Call to action: this research is open, civic, and for all of Earth.

\section{References}
[1] Placeholder for academic references.  
[2] Placeholder for community submissions.

\end{document}
